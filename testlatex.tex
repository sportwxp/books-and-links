\documentclass[12pt,twoside]{ctexart}
%\usepackage[T1]{fontenc}

 
\usepackage{CJK}
\usepackage{xcolor}
\usepackage{fancyhdr}
\usepackage{lipsum}
\usepackage{ifthen}
\usepackage{lastpage}
\usepackage[left]{lineno}
%设置页边距
\usepackage{geometry}
\geometry{a4paper,
 total={170mm,257mm},
 left=28mm,
 right=26mm,
 top=37mm,
 bottom=35mm}
 
 %设置字体,先添加family,再定义命令

\setCJKfamilyfont{fzxbs}{方正小标宋简体}            %方正小标宋简体
\newcommand{\bs}{\CJKfamily{fzxbs}}

\setCJKfamilyfont{fs}{仿宋_GB2312}                      %仿宋GB2312 
\newcommand{\fs}{\CJKfamily{fs}}    
                        
\setCJKfamilyfont{kai}{楷体_GB2312}                     %楷体GB2312 
\newcommand{\kai}{\CJKfamily{kai}}   

\setCJKfamilyfont{hei}{Hei}                     %楷体GB2312 
\newcommand{\hei}{\CJKfamily{hei}}   

\usepackage{fontspec}
\setCJKmainfont{仿宋_GB2312}

%设置字号
\usepackage{stix}
\newcommand{\yihao}{\fontsize{36.9pt}{\baselineskip}\selectfont} %标题
\newcommand{\erhao}{\fontsize{22pt}{\baselineskip}\selectfont}      %二号
\newcommand{\xiaoerhao}{\fontsize{18pt}{\baselineskip}\selectfont}  %小二号
\newcommand{\sanhao}{\fontsize{16pt}{\baselineskip}\selectfont}  %三号
\newcommand{\sihao}{\fontsize{14pt}{\baselineskip}\selectfont}%四号
\usepackage{type1cm}
\setlength{\parindent}{2em}
\linespread{1.43}%国军标要求22行每行28字

%设置页头和页尾
\newlength{\footlength}
\setlength{\footlength}{10pt}
\usepackage{amsmath,fancyhdr}
\pagestyle{fancy}
\fancyhf{}%清空head foot
\renewcommand{\headrulewidth}{0pt}
\fancyfoot[LE]{{\sihao \qquad --\hspace{\footlength}\thepage \hspace{\footlength}--}}
 \fancyfoot[RO]{{\sihao --\hspace{\footlength}\thepage \hspace{\footlength}--\qquad }}
 
 %版头
\newcommand{\bantou}[1][中央军委训练管理部办公厅]{\sanhao{\color{red}
\centering
{\bs \yihao{#1}}\par
\rule[17pt]{\linewidth}{0.5mm}\\[-23pt]}}

%函号
\newcommand{\hanhao}[1]{{
\hei \noindent 001\hfill {\fs 军训办函$\lbrbrak\space2020\space \rbrbrak$\space #1}\\
秘密$\bigstar{}$\ 5年\par \thispagestyle{empty}
\vspace{2em}}}

%标题
\newcommand{\biaoti}[1]{\centerline{\bs\erhao #1}\vspace{1em}}

\newcommand{\shoufang}[1]{{\noindent \kai  #1:}}


\newcommand{\fujian}[1]{\par \vspace{1em}
\hangindent=2.6\parindent
附件:1、《管理规定也有公共卫生等方面的丰富专业知识,以及将这些专业知识转化为新疗法和预防措》#1
}
 
 %发文单位
\newcommand{\danwei}[3]{
%\ifthenelse {\isodd{\value{page}} \or \lengthtest{therest < \bottomheight}}
	\ifthenelse {\isodd{\value{page}}}
       {\newpage %奇数页
       \vspace{3em} (此页无正文)
       \vspace{3em} \\ 
       \mbox{}\hspace{10em}
       \centerline{ #1} \\
       \centerline{2020年#2 月#3日}
       }{
       \ifthenelse{\inputlineno >16}{
       \newpage       %偶数页空间不够
       \newpage
       \vspace{3em} (此页无正文)
       \centerline{ #1} \\
       \centerline{2020年#2 月#3日}
	}
	\vspace{3em} \\ 
       \centerline{ #1} \\
       \centerline{2020年#2 月#3日}
}

\newcommand{\houji}[5]{
\vfill
\noindent \rule[-8pt]{\linewidth}{0.35mm}
{\sihao \fs  抄送:#1  \hfill (共印 #2 份)\\[-23pt]}
\rule{\linewidth}{0.25mm}\\[-9pt]
{\sihao \fs 承办单位:#3 \hfill 经办人:#4 \hfill 电话:#5\\[-23pt]}
\rule{\linewidth}{0.35mm}}
  





 


\begin{document}
\bantou[中央军委训练管理部办公厅]
\hanhao{87号}
\biaoti{关于什么什么的通知}
%\lipsum[1-2]
\shoufang{训练局}

二号方正大标宋(党口公文),二号方正小标宋(政口公文) 正文:三号仿宋字,一级标题三号黑体字、二级标题三号楷体字、三级标题仿宋字可加粗 行间距:设定页面上37mm,下35mm,左28mm,右26mm,每页22行,每行28字,行间距28磅(word),Covid-19造成了全球性危机。这场危机是对领导力的一场考验。由于并无对抗新型病原体的理想手段,因此各国被迫就如何应对做出艰难选择。而美国的领导人没有通过这场考验。他们把危机变成了悲剧。

此次失败触目惊心。根据约翰霍普金斯大学系统科学与工程中心(Johns Hopkins Center for Systems Science and Engineering)[1]数据,美国Covid-19患病人数和死亡人数都居于世界首位,远远超过中国等人口更多的国家。美国的病死率是加拿大的2倍多,比人口老龄化并因而拥有众多Covid-19高危人群的日本高出近50倍,甚至比越南等中低收入国家高出近2,000倍。Covid-19是一项巨大挑战,许多因素造成了其严峻程度。但如何应对是一个我们可以控制的因素,而美国的表现自疫情暴发就一直不佳。

美国遭遇这场危机时拥有巨大优势。除了巨大的生产能力,我们还有全世界称羡的生物医学研究体系。我们在公共卫生、卫生政策和基础生物学方面拥有强大专业能力,并且一直能够将这些专业能力转化为新疗法和预防措施。这些专业能力很多来自政府机构,可我们的领导人却选择无视甚至诋毁专家。

国家领导人应对疫情的措施一直不够充分。联邦政府基本上已经将疾病控制工作甩给了各州。各州州长的应对各不相同,原因主要不是党派政治,而是能力。但不论他们能力如何,州长都不拥有在联邦政府控制下的各种应对手段。联邦政府没有应用这些手段,而是削弱了它们。美国疾病控制与预防中心曾是世界领先的疾病防控组织,如今却遭受重创,并且在检测和政策方面遭遇惨败。美国国立卫生研究院在疫苗开发方面发挥了关键作用,但却被排除在政府的许多关键决策之外。美国食品药品管理局已经被可耻地政治化了[3],他们似
假如有其他人如此毫无顾忌地糟蹋生命和金钱,他们一定会受到法律惩罚。我们的领导人却声称他们的行为享有豁免权。但是,即将到来的选举赋予了我们审判其行为的权力。理性的人当然会对候选人的许多政治立场持不同意见。但真相本身既不持自由派观点,也不持保守派观点。在我们这个时代最严重的公共卫生危机面前,我们的领导人表现出了危险的无能。我们不应该纵容他们,不应该让这些人继续尸位素餐,而令美国的死亡

人数再多出成千上万人。
顶刊的悲歌
医学界的顶级期刊《新英格兰医学杂志》(the New England Journal of Medicine,NEJM)罕见地发表了一篇社论,谴责川政府的无能和失败,导致美国的新冠疫情变成一出悲剧,呼吁通过选举让川政府下台。
社论指出,美国原本拥有很强的生物医学研究系统,也有公共卫生等方面的丰富专业知识,以及将这些专业知识转化为新疗法和预防
\newlength{\bottomheight}
\setlength{\bottomheight}{7\baselineskip}
\the\bottomheight
\the\inputlineno

\fujian{1}
%\danwei{信息中心}{5}{1}
\houji{北京大学}{5}{综合室}{五线谱}{826154}

\end{document}